\chapter{Introduction to Python Programming}
\label{ch:python-intro}

Python is like a friendly dragon that helps you solve problems - powerful yet approachable! Let's embark on this exciting journey of learning Python programming.

\section{What Makes Python Special?}
\label{sec:python-special}

\begin{definition}[Python Programming Language]
    Python is a high-level, interpreted programming language known for its simplicity and readability. It emphasizes code readability with its notable use of significant whitespace.
\end{definition}

\begin{example}[Hello, World!]
    Your first Python program is as simple as:
    \begin{verbatim}
    print("Hello, World!")
    \end{verbatim}
    That's it! No complicated setup, no extra symbols - just pure simplicity.
\end{example}

\section{Python Building Blocks}
\label{sec:building-blocks}

\subsection{Variables and Data Types}
\label{subsec:variables}

Think of variables as magical containers that can hold different types of data.

\begin{definition}[Variables]
    Variables are named storage locations that can hold data. In Python, you don't need to declare the type - Python figures it out automatically!
\end{definition}

\begin{example}[Variable Usage]
    \begin{verbatim}
    name = "Alice"        # A string
    age = 25             # An integer
    height = 1.75        # A float
    is_student = True    # A boolean
    \end{verbatim}
\end{example}

\begin{remark}
    Unlike many other programming languages, Python variables don't need type declarations. This is called dynamic typing!
\end{remark}

\section{Control Flow: Making Decisions}
\label{sec:control-flow}

\subsection{If Statements}
\label{subsec:if-statements}

\begin{example}[Simple Decision Making]
    \begin{verbatim}
    age = 18
    if age >= 18:
        print("You can vote!")
    else:
        print("Wait a few more years.")
    \end{verbatim}
\end{example}

\section{Loops: Doing Things Repeatedly}
\label{sec:loops}

\begin{definition}[Loops]
    Loops are structures that allow you to repeat a block of code multiple times. Python has two main types: 'for' and 'while' loops.
\end{definition}

\begin{example}[For Loop]
    \begin{verbatim}
    # Print numbers from 1 to 5
    for i in range(1, 6):
        print(i)
    \end{verbatim}
\end{example}

\begin{exercise}[Loop Practice]
    Write a program that prints the first 10 even numbers.
\end{exercise}

\begin{solution}
    Here's one way to solve it:
    \begin{verbatim}
    for i in range(2, 21, 2):
        print(i)
    \end{verbatim}
\end{solution}

\section{Functions: Your Own Commands}
\label{sec:functions}

\begin{definition}[Functions]
    Functions are reusable blocks of code that perform specific tasks. They're like your personal recipes for solving problems!
\end{definition}

\begin{example}[Simple Function]
    \begin{verbatim}
    def greet(name):
        return f"Hello, {name}!"
    
    # Using the function
    message = greet("Python Learner")
    print(message)
    \end{verbatim}
\end{example}

\begin{note}
    Python functions can return multiple values! This is a powerful feature not found in many other programming languages.
\end{note}

\section{Practical Projects}
\label{sec:projects}

Let's put everything together with some fun projects!

\begin{exercise}[Temperature Converter]
    Create a function that converts temperatures between Celsius and Fahrenheit.
\end{exercise}

\begin{solution}
    \begin{verbatim}
    def celsius_to_fahrenheit(celsius):
        return (celsius * 9/5) + 32
    
    def fahrenheit_to_celsius(fahrenheit):
        return (fahrenheit - 32) * 5/9
    
    # Test the functions
    print(celsius_to_fahrenheit(0))  # 32.0
    print(fahrenheit_to_celsius(32)) # 0.0
    \end{verbatim}
\end{solution}